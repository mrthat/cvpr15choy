\documentclass[10pt,twocolumn,letterpaper]{article}

\usepackage{cvpr}
\usepackage{times}
\usepackage{epsfig}
\usepackage{graphicx}
\usepackage{amsmath}
\usepackage{amssymb}
\usepackage{caption}
\usepackage{verbatim}
\usepackage{subcaption}
\usepackage{algorithm2e}
\usepackage{rotating}
\usepackage[space]{grffile}
\usepackage[font=small,skip=0pt]{caption}
%\DeclareMathOperator{\Tr}{Tr}
% Include other packages here, before hyperref.

% If you comment hyperref and then uncomment it, you should delete
% egpaper.aux before re-running latex.  (Or just hit 'q' on the first latex
% run, let it finish, and you should be clear).
\usepackage[breaklinks=true,bookmarks=false]{hyperref}
\graphicspath{ {figures/} }
\cvprfinalcopy % *** Uncomment this line for the final submission

\def\cvprPaperID{2349} % *** Enter the CVPR Paper ID here
\def\httilde{\mbox{\tt\raisebox{-.5ex}{\symbol{126}}}}

% custom commands
\newcommand{\scream}[1]{{\color{red} \bf *** #1 ***}}

% Pages are numbered in submission mode, and unnumbered in camera-ready
%\ifcvprfinal\pagestyle{empty}\fi
\setcounter{page}{1}
\begin{document}

%%%%%%%%% TITLE
\title{Enriching Object Detection with 2D-3D Registration\\and Continuous Viewpoint Estimation}
%\title{Enrich Object Detection : 2D-3D registration and continuous viewpoint estimation}

\author{Christopher Bongsoo Choy\textsuperscript{\dag}, Michael Stark\textsuperscript{\dag \ddag}, Sam Corbett-Davies\textsuperscript{\dag}, Silvio Savarese\textsuperscript{\dag}\\
	\textsuperscript{\dag}Stanford University, \textsuperscript{\ddag}Max Planck Institute for Informatics\\
	{\tt\small \{chrischoy, scorbett, ssilvio\}@stanford.edu, \textsuperscript{\ddag}stark@mpi-inf.mpg.de }
% For a paper whose authors are all at the same institution,
% omit the following lines up until the closing ``}''.
% Additional authors and addresses can be added with ``\and'',
% just like the second author.
% To save space, use either the email address or home page, not both
}

\maketitle
%\thispagestyle{empty}

%%%%%%%%% ABSTRACT
\begin{abstract}
A large body of recent work on object detection has focused on exploiting 3D CAD
model databases to improve detection performance. Many of these approaches work
by aligning exact 3D models to images using templates generated from renderings
of the 3D models at a set of discrete viewpoints. However, the training
procedures for these approaches are computationally expensive and require
gigabytes of memory and storage, while the viewpoint discretization hampers
pose estimation performance.

We propose an efficient method for synthesizing templates from 3D models that
runs on the fly -- that is, it quickly produces detectors for an arbitrary
viewpoint of a 3D model without expensive dataset-dependent training or template
storage. Given a 3D model and an arbitrary continuous detection viewpoint, our
method synthesizes a discriminative template by extracting features from a
rendered view of the object and decorrelating spatial dependences among the
features. Our decorrelation procedure relies on a gradient-based algorithm that
is more numerically stable than standard decomposition-based procedures, and we
efficiently search for candidate detections by computing FFT-based template
convolutions. Due to the speed of our template synthesis procedure, we are able
to perform joint optimization of scale, translation, continuous rotation, and
focal length using Metropolis-Hastings algorithm. We provide an efficient GPU
implementation of our algorithm, and we validate its performance on 3D Object
Classes and PASCAL3D+ datasets.

%     Using 3D model to detect and register the model to RGB image has been a
%     growing field of study. Recently, Aubry \etal \cite{Aubry14} and J. Lim
%     \etal \cite{Lim14} tried to estimate pose and align exact CAD model to an
%     image using part-based templates. Malisiewicz \etal \cite{Malisiewicz11}
%     also addresses the issue by transferring metadata after detection. Hejrati
%     \etal \cite{Hejrati14} uses template synthesis to detect an object. All
%     these approaches require extensive training, a lot of memory space. Also,
%     the training based approach requires defining fixed viewpoints which limit
%     themselves.  To overcome these difficulties, we propose an efficient and
%     fast way to synthesize and validate templates using on the fly realistic
%     rendering that can jointly optimize scale, translation, rotation and focal
%     length continuously. We render an object and extract features and
%     decorrelate spatial dependencies within them on the fly to make a
%     discriminative template. To speed up the decorrelation procedure, we adopt
%     a gradient based algorithm which is numerically more stable than
%     decomposition procedure. Finally, we efficiently search for candidate
%     match by computing convolution using FFT and jointly optimize to match the
%     object accurately. To speed up further, we propose an efficient GPU
%     implementation and tested on PASCAL images.

%   In contrast to other mid-level patch based 3D matching methods which
%   requires extensive training and more than 10Gb of memory space and physical
%   storage, ours require no training nor saving the templates thus enabling it
%   to run on any personal computer.  To overcome these difficulties, we propose
%   an efficient and fast way to synthesize and validate templates using on the
%   fly realistic rendering that can jointly optimize scale, translation,
%   rotation and focal length continuously. We render an object and extract
%   features and decorrelate spatial dependencies within them on the fly to make
%   a discriminative template. To speed up the decorrelation procedure, we adopt
%   a gradient based algorithm which is numerically more stable than
%   decomposition procedure. Finally, we efficiently search for candidate match
%   by computing convolution using FFT and jointly optimize to match the object
%   accurately. To speed up further, we propose an efficient GPU implementation
%   and tested on PASCAL images.

\end{abstract}

%%%%%%%%% BODY TEXT
\section{Introduction}
\label{sec:intro}
\begin{figure}[t]
  \centering
  % \includegraphics[width=0.9\linewidth]{schematics} 
  \includegraphics[width=0.99\linewidth]{front} % \\[-5pt]
  % (a)\\[0pt]
  % \setlength\tabcolsep{0pt}
  % \begin{tabular}{cc}
  %     \includegraphics[width=0.45\linewidth]{car/orig2} & 
  %     \includegraphics[width=0.45\linewidth]{car/overlay2}
  %  %   \begin{tabular}{cc}
  %  %       \includegraphics[width=0.16\linewidth]{cad_car/2001-2004 Ford Ranger Edge} & 
  %  %       \includegraphics[width=0.16\linewidth]{cad_car/2002 Dodge Ram FedEx} \\ 
  %  %       \includegraphics[width=0.16\linewidth]{cad_car/2007-Nissan-Versa-_-Tiida-SL} & 
  %  %       \includegraphics[width=0.16\linewidth]{cad_car/2008-Jeep-Cherokee} \\ 
  %  %       \includegraphics[width=0.16\linewidth]{cad_car/2009 Toyota Cargo} & 
  %  %       \includegraphics[width=0.16\linewidth]{cad_car/2010-Chevrolet-Aveo5-LT} \\ 
  %  %       \includegraphics[width=0.16\linewidth]{cad_car/2012-Citroen-DS4}  
  %  %   \end{tabular}
  % \end{tabular}\\
  % (b)
  \caption{Using a database of 3D CAD models, we generate NZ-WHO
    templates which can be used to either detect objects directly or
    enrich the output of an existing detector with high-quality,
    continuous pose and 3D CAD model exemplar.}
    % initialize pose of an object in the bounding box from other
    % detectors. Once initialization is given, we use NZ-WHO to propose
    % and validate plausible pose on-the-fly and
    % further tune the translation, 3D viewpoint and focal length continuously.}
  % \vspace{-1em}
  \label{fig:front}
\end{figure}
 
% Object detection task has long been about predicting rectangular bounding boxes around the object. In this workthere has been significant many powerful object detectors such as Convolutional Neural Network or Part-Based Models \cite{Girshick14, Felzenszwalb10, Pepik12} 

Current approaches to object class detection have reached a remarkable
level of performance in 2D bounding box
localization~\cite{pascal12, Felzenszwalb10, Overfeat, Krizhevsky12, Girshick14}, due to their
ability to generalize across differences in object appearance,
lighting, and viewpoint. While this generalization
ability of these methods is beneficial for robustness, it also limits
the level of detail that these detectors can deliver as an output.

As a consequence, there has been increased interest in multi-view object
recognition, where viewpoint estimates are provided by
detectors in addition to 2D bounding boxes. Several attempts have been
made to estimate viewpoint along with object class detections,
implemented as extensions to existing detectors, such as the implicit
shape models~\cite{sun10eccv}, the constellation model~\cite{stark10bmvc}, or the
deformable part model~\cite{Felzenszwalb10, gu10eccv,Xiang12,Pepik12,Fidler12,Hejrati14}.

Recently, an even higher level of geometric detail was reached in the
form of aligning
3D CAD model instances to real world test images~\cite{Aubry14,
  Lim14,Kholgade14, Chen13, Kostas14}. Interestingly, the problem of matching 3D models to 2D
images has been explored since the early days of computer
vision~\cite{Lowe87}, but had largely been neglected in recent years
in favor of 2D detectors based on robust local features and
statistical learning techniques. Now, this problem is being revisited
mostly due to two reasons: {\em (i)} the availability of 3D CAD models
for many object classes and {\em (ii)} the availability of robust
image matching techniques.
%
% performance object detectors usually generalizes very well. It
% can handle various intraclass variability, viewpoint change and
% occlusion. However due to this high generalization, sometimes it is
% very difficult to get accurate pose, category or 2D-3D matching. Our
% method, specialized in giving such accurate information, can be
% applied on any generic object detectors to give high quality metadata.
% %
% Matching 3D CAD model to an image has been a classical problem that
% dates back to eighties . Lowe \etal , the vision
% community instead focused on the object detection tasks that estimates
% a rectangular bounding box around an object\cite{Felzenszwalb10,
%   Girshick14}. 
%
%In addition, there has been a growing emand for accurate 2D-3D
%matching \cite{Kholgade14, Chen13, Kostas14} for image manipulation,
%editing, and data augmentation in the computer graphics community.
%
% Recently, with the powerful computing capabilities and a large
% collection of CAD models, the 2D-3D matching (registration) problem
% gets more and more approachable but due to inherent difficulty and
% limited performance of existing methods, many applications still
% require supervision to achieve 2D-3D matching.
% In medical community, registering 3D scan data to 2D image enabled various minimally invasive surgeries \cite{Markelj12}.
%
% For category-level matching, \cite{Xiang12, Hejrati14, Pepik12} used
% 3D parts or aspect to represent underlying category-level 3D geometry
% of the object. With the advent of large public CAD datasets
% \cite{Trimble, Yobi}, the instance-level object detection and 2D-3D
% registration became possible. \cite{Aubry14, Lim14} used large
% collection of CAD models to faithfully align 3D CAD model to an image
% by rendering CAD models and creating 2D templates for various
% viewpoints.

For (i), recent approaches to 2D-3D matching~\cite{Aubry14, Lim14}
typically rely on a
collection of 3D exemplar models, which they render from a
large number of viewpoints. The resulting artificial images are then
used to train exemplar models that can be matched to a real-world
image at test time.
%
For (ii), it has been realized that template-based exemplar detectors
based on HOG~\cite{Dalal05} features can be trained analytically, by
replacing the standard SVM with an LDA classifier
(E-LDA)~\cite{Hariharan12}, resulting in a whitened feature
representation, termed WHO. As a result, it is feasible to train
hundreds of thousands of mid-level patch detectors for
recognition~\cite{Aubry14}.
%
Unfortunately, the performance of WHO relies crucially on an
additional calibration step that equalizes the detection scores of
independently trained exemplar models. Since this step involves costly
mining of hard negative examples~\cite{Dalal05,Felzenszwalb10} on a
validation set, it constitutes the major computational bottleneck of
WHO, limiting its scalability.

In this paper, we propose a novel method for 2D-3D alignment of
exemplar CAD models to real-world images that circumvents the need for
calibration, and greatly enhances the scalability of WHO. As a
result, we can render novel views and train corresponding exemplar
models {\em on the fly}, without the need for offline processing. As a
by-product, we can formulate the alignment problem as a parameter
search in a continuous pose space, consisting of yaw, pitch and roll, resulting in truly continuous pose estimates, which
we implement using MCMC sampling.
%  require a large number of rendering and 2D
% templates for each of the CAD model with specific viewpoint. Since
% there are so many templates that it becomes almost impossible to get
% all the true positive image from training data. Instead \cite{Aubry14,
%   Lim14} used whitening as a preprocessing stage \cite{Hariharan12} to
% make templates more discriminative. Then calibration is done on a
% small training dataset.
%
% Although whitening to create template is faster than some of
% instance-level matching algorithms such as training Exemplar-SVM
% \cite{Malisiewicz11} or complicated models such as LEVAN
% \cite{Divvala14}, whitening and calibration large number of templates
% takes very long time. Also, once the fixed discrete viewpoint
% templates have been made and calibrated, it is difficult to fine-tune
% the prediction made using the large collection of templates.

Our paper makes the following specific contributions.

First, we present a novel method for training exemplar models from
rendered 3D CAD data {\em on the fly}, enabled by a novel variant of
WHO, termed NZ-WHO, and making efficient use of the specific
characteristics of rendered images. To our knowledge, our method
constitutes the first attempt to simultaneously render and train
exemplar detectors on the fly.
% In our paper, we present Non-Zero Whiten Histograms of Orientations
% (NZ-WHO) to dramatically speed up the template generation and get rid
% of time consuming calibration and yet achieves performance on par with
% calibrated templates. First, we will present NZ-WHO is more
% discriminative than WHO and introduce Conjugate Gradient method to
% speed up the whitening and create a high resolution template in just
% 80 milliseconds. Also, using this fast NZ-WHO template generation, we
% present \textbf{on-the-fly} template generation to fine tune an
% initial pose estimation to continuously refine translation, viewpoints
% (yaw, roll, pitch) and focal length.

Second, we show to enrich the output of an existing object class
detector, such as the DPM~\cite{Felzenszwalb10} or the
R-CNN~\cite{Girshick14} with additional 3D information, applying our
method to candidate detections provided by the respective detector. As
a result, we can refine the original detections with additional
estimates of 3D continuous pose and a 3D CAD model exemplar.

And third, we give an in-depth experimental study that demonstrates
the effectiveness of our approach on a standard benchmark for
object detection and viewpoint estimation~\cite{Xiang14}. 

\begin{comment}
\scream{Are we keeping the paragraphs from here...}

outperforming previous results on this dataset by considerable
margins.



%% Our contribution
With this growing usage of WHO template and Computer Graphics
rendering, we decided to analyze WHO feature
[Find More Usage of WHO and CG rendering \cite{Pepik12}]




Our pipeline is simple, fast only requires CAD models and no
validation images for calibration. This can be used for post
processing stage of generic object detection. Our pipeline requires
only few minutes to make templates for rough viewpoints and 


We propose a simple method to register the most similar CAD model to
an image and fine-tune viewpoint continuously without discriminative
part search, training nor calibration. Our method is simple, fast and
light-weight in terms of memory so that it can be easily applied on
top of generic object detectors as a final stage of detection or can
be used as a standalone detector. 



% Although our method is geared toward instance-level 2D-3D matching, we evaluated the performance of the templates on PASCAL 2007 and PASCAL 2012 and PASCAL 3D dataset to measure the performance. 

Our tuning stage not only refine position, scale and viewpoint, but
also focal length as well to literally "fine-tune" the initial
proposal. Though our 2D-3D matching method can be used as a
rudimentary object detector, our simple tuning stage can be applied to
any generic object detector to achieve state-of-the-art viewpoint
estimation method.


Finally, we show several applications using our method. Since we have
metadata such as CAD model and geometry, we reconstruct the object
using 2D-3D matching. Also, since we do not need training image, we
show that our 2D-3D matching performs better than standard detectors
in categories where training data is insufficient. We then use our method for
image query to pull out the closest CAD model in a bank of CAD models.


Finally, unlike several CAD model alignment methods that uses  WHO
\cite{Aubry13, Aubry14, Lim14}, we propose a novel pipeline that does
not require calibration which is infeasible in continuous viewpoint
estimation. We propose a simple add-on stage that can enrich object
detections with high-quality 2D-3D alignment. The pipeline is simple
and light weight so this can be easily adapted to be used for any
generic object detectors. Also to facilitate future work, all our
alignment result on PASCAL12 and code are publicly available online.




% Recently, matching a 3D CAD model to an image, 2D-3D matching (registration), is gaining great momentum in various fields of study. 2D-3D matching using Structure from Motion \cite{Sattler11} uses existing 3D model to match the model to a query image. Also image manipulation using 3D CAD models has proven its potential. Kholgade \etal \cite{Kholgade14} and Chen \etal \cite{Chen13} require human input to match 3D models to an image to created realistic morphing or manipulation of the registered object. However, all these application requies human input. In this paper, we propose a novel on-the-fly template generation algorithm that can generate template quickly and validate 2D-3D pose and focal length continuously.

% Our code has been made publicly online \cite{Choy14} so that our algorithm can be used as a pipeline for 2D-3D registration after general purpose object detectors.

\scream{... to here?}
\end{comment}





%
\section{Related Work}
\label{sec:related}
%\subsection{Object Detection and Viewpoint Estimation}
%
Modern object detectors generalize very well, handling intraclass variability,
occlusion, truncation and viewpoint changes \cite{Felzenszwalb10, Girshick14}.
However, this generalization comes at the cost of fine-grained information,
including accurate 3D pose and object sub-category recognition. Such methods
typically produce bounding box detection hypotheses, with little further
information.

Many methods have attempted to move object detection towards richer outputs,
especially by jointly performing detection and pose estimation \cite{Pepik12,
Xiang12, Fidler12, Xiang14, Hejrati14, Aubry14, Lim14}. To achieve this,
\cite{Xiang12, Hejrati14, Fidler12} use 3D representations that deform as
viewpoint changes and \cite{Pepik12} uses geometric constraints to regularize 2D
appearance models.

The methods above perform discrete pose estimation, quantizing the viewing
sphere into a number of poses and selecting the best one during inference.
Fine-grained pose estimators, in contrast, can infer continuous (or arbitrarily
fine-grained) poses. One such method from \cite{Zia13} aligns a 3D deformable
part-based wireframe model with input images to accurately predict object poses.

More recently, \cite{Aubry14, Lim14} made progress in joint instance-level
object detection and pose estimation. To estimate pose they use synthetic
renderings of CAD models to learn discriminative mid-level patches.
\cite{Aubry14} calibrates these patches on a small set of real images, while
\cite{Lim14} presents a method for learning the relative discriminativeness of
the patches.

% \textcolor{red}{Again, should we highlight the differences between these methods and ours?}

% specialized in giving such accurate information, can be applied on any generic
% object detectors to give high quality metadata.  Many Since the WHO does not
% require extensive training such as \cite{Felzenszwalb10, Malisiewicz11,
% Girshick14}, WHO is gaining momentum in 2D-3D matching. \cite{Aubry13,
% Aubry14, Lim14} combined WHO with synthetic rendering to jointly detect and
% estimate pose by making mid-level patches using WHO. Using this WSuch approach
% has an advantage over generic detections. First, it is easy to trace the
% detection to specific template. The template also has The rendering has
% corresponding CAD model and viewpoint thus enabling joint detection and
% viewpoint estimation.  To deal with textureless background \cite{Aubry13}
% whitens HOG feature and then zero out the WHO feature where there is no
% texture. But centering a HOG feature where there is no edge creates strong
% negative edges. We found out that centering only non-zero cells help detection
% but whitening all cell makes features to leak out to texture-less regions.
% With increasing popularity of rendering image, whitening, traditional signal
% processing technique, as a preprocessing stage has been widely used also in
% computer vision community.  Many object detectors \cite{Felzenszwalb10,
% Malisiewicz11} \subsection{Mid-level Patches for 2D-3D matching} FPM
% introduction : per instance matching  Seeing 3D chair : mid-level patches
  
%\begin{comment}
%  \subsection{Contributions}
 % \begin{enumerate} \itemsep1pt \parskip0pt \parsep0pt
  %  \item Introduces Non-Zero Whitened Histograms of Orientations (NZ-WHO) for real-time template generation
   % \item Continuous viewpoint and focal length tuning using on-the-fly template generation
%    \item Jointly optimizes translation, 3D viewpoint, CAD models and focal length
%     \begin{comment}
%       \item HOG Whitening procedure that do not require calibration stage
%     \end{comment}
%  \end{enumerate}
% \end{comment}
% All the code and CAD models is publicly available online \cite{Choy14} (currently not disclosed for anonymous review, visualizations are available in the supplementary material).



\section{Approach Overview}
\label{sec:nz-who}
% \begin{comment}
% Our method attempts to match CAD models to 2D images, determining the best
% model within a class, its pose and scale, and the focal length of the camera
% used to produce the image. These parameters are first coarsely estimated using
% an ensemble of templates generated offline, before they are refined using
% MCMC by generating new detectors on the fly.
% 
% % The basic detection method underlying our approach is WHO template matching.
% % However, we propose a new WHO variant called Non-Zero Whitened Histograms of
% % Orientations (NZ-WHO), where the variant of WHO templates are made from 
% % synthetic rendering to with 
% 
% Our method In addition to the library of CAD models that will be matched to the query
% images, our method requires a feature statistics which can be collected once and for all.
% 
% % images, our method requires a collection of random natural images. Offline, we
% % compute HOG features of these images and record their mean and spatial
% % autocovariance. The images are not used during testing, only the statistics
% % generated from them.
% 
% Notably, we do not require any labelled training images of the objects to be
% detected. Our templates are generated exclusively by rendering CAD models from
% our library. From these renderings we produce novel HOG-like features we call
% Non-Zero Whitened Histograms of Orientations (NZ-WHO), which use the natural
% image statistics found offline to whiten a standard HOG template, making it
% more discriminative.
% 
% The matching proceeds in two steps: First, we apply a bank of pre-trained
% detectors to the image, corresponding to a set of discrete viewpoints, scales,
% and object models. This provides a rough estimate of these parameters. Second,
% we refine these parameters using MCMC, where each candidate is evaluated by
% rendering a new whitened template on the fly. 
% 
% Our ensemble of NZ-WHO templates can work as a detector but our algorithm 
% can be initialized using proposal bounding boxes from
% state-of-the-art object detectors \cite{Felzenszwalb10,Girshick14} before
% executing the two steps above. In this scenario, our approach provides a means
% to augment standard bounding box detectors with more interesting 3D
% information.
% \end{comment}

% \begin{comment}

Our method has two modes of operation.

{\em (i)} it can be run in isolation, as a sliding window object
detector, similar in spirit to exemplar SVM~\cite{Malisiewicz11}. In
that case, it can not only provide a detection bounding box but also
meta-data such as viewpoint or 3D CAD model exemplar.
As we show in our experiments, our method in isolation delivers
performance that is on par with state-of-the-art for the task of
object class detection and viewpoint estimation while at the same time
being much faster to train and requiring no training images(Sect.~\ref{sec:exp_iso}).

{\em (ii)} it can be used to enrich the detections of another object
class detector that proposes candidate regions that can then be
refined by our method. This second mode of operation constitutes the
strength of our method, as we will show in our experiments
(Sect.~\ref{sec:experiments}).

Both modes of operation rely on the succession of two steps:
on-the-fly generation of exemplar templates, based on a novel whitened
feature representation termed NZ-WHO (Sect.~\ref{sec:approach}), and
pose fine-tuning using MCMC (Sect.~\ref{sec:fine}).




% with rich annotations. 
% mode uses the common path that goes from initializtion to fine
% tuning~\ref{fig:front}.  

% For both modes, they require pose initialization by an ensemble of
% NZ-WHO templates. We will explain the NZ-WHO generation pipeline in
% Sect.~\ref{sec:approach}. First stage of pipeline starts from
% rendering an viewpoint specific template in
% Sect.~\ref{sec:rendering}. Then we will go over our novel combination
% of rendering with Conjugate Gradient method to produce a
% discriminative NZ-WHO template quickly in subsequent sections from
% Sect.~\ref{sec:who} to Sect.~\ref{sec:fastwhiten}.

% Once we make an ensemble of NZ-WHO templates, we use it as a
% initialization for subsequent fine tuning stage where our model
% further tunes the 2D-3D matching (Sect.~\ref{sec:fine}).

% For mode (i), we start from detection bounding boxes that has little
% to no metadata and run initialization stage and tuning stage.


% For mode (ii), we start directly from initialization by applying our
% ensemble of NZ-WHO templates as an ensemble of exemplar detectors.

% To validate our approache, in Sect.~\ref{sec:experiments}, we (a)
% compare the performance of NZ-WHO with other variants of WHO, (b) run
% the mode (i) on PASCAL 3D dataset~\cite{Xiang14} and (c) run the mode
% (ii) on 3D Object dataset~\cite{Savarese07}.









% % Might use the following in introduction
% \scream{More like an intro than an overview:
% Our method is in line with \cite{Aubry14, Lim14} in the sense that we
% use only renderings, not real images. However, rather than using fixed
% sized mid-level patches, 

% \scream{The following part is too harsh}
% we use whole objects to make root
% templates. By using root templates, there is no need to learn a
% discriminative patch dictionary \cite{Aubry13, Aubry14, Lim14}, calibrate the patches \cite{Aubry14, Lim14}
% nor add inference stage that uses spatial configuration \cite{Aubry14, Lim14}. 

% However, whitening the whole template is prohibitive for several reasons. All templates have
% different aspect ratios making it difficult to generate the covariance
% matrix $\Sigma$. Since the root templates are larger than mid-level
% patches, a very large auto-covariance $\Gamma$ is required. Lastly,
% time complexity of decorrelating a template with $n$ cells increases
% as $O(n^3)$, resulting in slow template generation and large residuals
% due to numerical errors accumulated while decomposing the covariance
% matrix $\Sigma$.

% To overcome these difficulties, we propose the Non-Zero Whitened
% Histograms of Orientations (NZ-WHO) with iterative Conjugate Gradient to
% generate templates in real time and enrich rectangular boxes with high
% quality 2D-3D matching. The 2D-3D matching provides a CAD model,
% sub-category, rendering, depth, viewpoint and focal length.

% Our method can work as a standalone 2D-3D matching algorithm, or even
% as a standalone ensemble of exemplar-based object detectors (Figure
% \ref{fig:front}) but its real power comes from combining generic
% object detectors and complement rectangular bounding boxes. Our method
% is simple in design; is several orders of magnitude faster than
% traditional method; requires only CAD models; does not need neither images nor
% calibration, yet is powerful in giving high-quality metadata. These
% properties make our method ideal for complementing object detections. 

% To complement detection bounding boxes, we first search possible 2D-3D
% matching by running cached discrete viewpoint templates to initialize MCMC
% continuous fine tuning stage. Then it refines continuous pose, focal length and
% discrete CAD model index, scale translation. Since it requires iterations of
% proposal and validation, we
% present real time analysis of NZ-WHO template generation to show that
% this can be done on-the-fly. The fine tuning stage process jointly
% optimizes translation, yaw, roll, pitch, scale, and focal length
% continuously and also searches for different CAD models.}


\section{On-the-Fly Template Generation}
\label{sec:approach}
In this section, we describe our approach to generating 3D
exemplar-based templates on-the-fly. It is based on 3D CAD model
rendering (Sect.~\ref{sec:rendering}) followed by feature extraction
and applying a whitening procedure. Based on the original whitening
formulation (Sect.~\ref{sec:who}), we propose three novel extensions that
enable the application of whitening to the on-the-fly setting.
%
First, we adapt the whitening to the specific case of rendered
images (Sect.~\ref{sec:nzwho}). Second, we show how to speed up the
whitening by two orders of magnitudes for high-resolution templates
(Sect.~\ref{sec:fastwhiten}). And third, we improve the runtime of our
3D exemplar template detectors also at test time, by performing
convolutions in the frequency domain (Sect.~\ref{sec:fft}).


\subsection{Rendering}
\label{sec:rendering}

We used an off-the-shelf rendering engine to generate a realistic rendering
and a depth map. The CAD models we used contain texture and material
information. These make the rendering more realistic and allows us to transfer
natural image statistics to rendered images. Along
with a rendering, we extracted a depth map which can be used for
reconstruction\ref{fig:rendering}.

To handle intraclass and viewpoint variability, we used various CAD models and
made renderings of these CAD models from different viewpoints.  Note that we
continuously vary yaw, pitch, roll and the focal length as well so that the
final fine tuning stage can produce accurate viewpoint estimations
(Sect.~\ref{sec:fine}).

\begin{figure}[t]
  \begin{center}
    \includegraphics[width=0.4\linewidth]{rendering}
    \includegraphics[width=0.4\linewidth]{depth}
  \end{center}
  \caption{An example rendering and depth image from renderer.}
  \label{fig:rendering}
\end{figure}


\subsection{Whitened Histograms of Orientations (WHO)}
\label{sec:who}
Our technique for rendering and generating exemplar template detectors
on-the-fly is drawing from recent work. Hariharan \etal. introduced Whitened
Histograms of Orientations (WHO), which uses feature statistics from natural
images to create a large number of classifiers analytically using Linear
Discriminant Analysis(LDA) rather than training SVM
classifiers\cite{Hariharan12}. The confidence score for data $x_i$ can be
defined as $ S(x_i) = w_{x_t}^T x_i$ where $w_{x_t} = \Sigma^{-1} (x_t - \mu)$
is an LDA classifier for a template $x_t$. Since collecting covariance matrices
for all possible template shape is intractable, \cite{Hariharan12} assumed
Wide-Sense Stationarity(WSS) of HOG features and generated a covariance
$\Sigma$ from autocovariance $\Gamma$ collected on a large collection of
natural images. i.e. for a 31 dim HOG feature $x_t$ at location $t$

\begin{align}
    E[x_t] & = \mu\\
    E[(x_{t} - \mu ) (x_{\tau} - \mu)] & = E[(x_0 - \mu)(x_{\tau - t} - \mu)]
\end{align}

In this paper, we followed their method to generate $\Sigma$.


% \begin{align}
%     S(x_i) = w_{x_t}^T x_i\\
%     w_{x_t} = \Sigma^{-1} (x_t - \mu)
% \end{align}

% The advantage of the approach is twofold: (a) you do not need a large number of
% training and hard negative mining \cite{Felzenszwalb10} (b) it takes a fraction
% of time generate (train) a discriminative template.

% templates, making them more discriminative. Whitening is a common signal
% processing operation for decorrelating a set of random variables
% \cite{Martinsson05, Belouchrani00}. More formally, suppose that we have a
% $k$-dimensional random variable $X \in \mathcal{R}^k$ with $Cov(X)=\Sigma$. By
% whitening the signal,
% 
% \begin{equation}
% \Phi(X)=\Sigma^{-\frac{1}{2}}(X - E[X]) \label{eq:whitening}
% \end{equation}
% 
% we remove 2nd order correlation between all features. Since collecting covariance
% matrix for every template shape is expensive, \cite{Hariharan12} proposed an easy
% way to synthesize
% covariance $\Sigma$ from autocovariance $\Gamma$ and we followed their
% method to generate $\Sigma$.


% More formally, suppose that we have a
% $k$-dimensional random variable $X \in \mathcal{R}^k$ with $Cov(X)=\Sigma$. By
% whitening the signal,
% \begin{equation}
% \tilde{X}=\Sigma^{-\frac{1}{2}}(X - E[X]) \label{eq:whitening},
% \end{equation}

% is independent of its place in the image, and the autocovariance of cells
% depends only on their relative location.

% The whitening and LDA have the same
% formulation when we make a decision boundary $w = \Sigma^{-1}(\mu_+ - \mu_0)$
% where $\mu_+$ is the features from a class (LDA) or a signal to whiten
% (whitening), $\mu_0$ is the negative class center (LDA) or the signal mean
% (whitening) and $\Sigma$ is the covariance matrix of classes (LDA) or
% the correlation within signal (whitening).

% In one dimension, let $x(u)$ be the feature at location $u$. Then WSS states
% that $\mathbb{E}\left[x(u)\right] = \mu$ for all $u$, and
% \begin{equation}
% \textrm{cov}_x(u,v) = \textrm{cov}_x(0, v-u) = \Gamma(v-u),
% \end{equation} where $\Gamma$ is the autocovariance. For simplicity, we describe the 1D case but this can be easily extended to 2D spatial autocovariance.
% 
% Therefore, assuming WSS allows the covariance matrix of templates \emph{of any
% size} to be synthesized from the autocovariance matrix using a simple lookup.

% However, the 
% We propose Non-Zero Whitened Histograms of Orientations (NZ-WHO), that are both
% more discriminative than WHO, and can be generated several orders of magnitude
% faster (in 70ms compared to several seconds for WHO). This speed up means we
% can generate templates on the fly, allowing our approach to evaluate arbitrary
% viewpoint hypotheses.

% Our method requires generating a covariance matrix of non-zero cells 

\subsection{Whitening Synthesized Templates and Non-Zero WHO}
\label{sec:nzwho}
Our first improvement is `Non-Zero' whitening. When synthesizing detection
templates from rendered images, a common problem is how to handle the
background. If the model is rendered over a natural image background, gradients
in the background will be incorporated into the discriminative template.

Alternatively, if the background is left textureless (see
Fig.~\ref{fig:rendering}), whitening the resulting HOG template
, strong negative weights are introduced in the textureless region (by
subtracting the mean $\mu$) as seen in Fig.~\ref{fig:whocomparison}.
This could result in positive matches being suppressed due to spurious
background gradients.

NZ-WHO removes these artifacts so that the background has no effect on the
template response. Let a vectorized HOG features of a rendering image $x = [
\begin{array}{ccccc}x_1^T & x_2^T & \cdots & x_n^T\end{array} ]^T \in
\mathcal{R}^{nd}$ where $x_i$ is the $i$th HOG cell feature and $n$ is the
number HOG cells and $d$ is the dimension of HOG feature. We create a new
vector $\bar{x}$ which contains only the non-zero HOG cell features of $x$. To be
specific, let $I_d\in \mathcal{R}^{d\times d}$ be the identity matrix, $\bar{n}$ be
the number of non-zero HOG cells, and a matrix $S\in \mathcal{R}^{\bar{n}d \times
    nd}$ be the masking matrix that selects non-zero HOG cells. For instance, a
template has  $n=3, \bar{n}=2$, and only the second HOG cell has $0$ norm, then

\begin{align}
    S = \left[ \begin{array}{ccc}
        I_d & \mathbf{0} & \mathbf{0}\\
        \mathbf{0} & \mathbf{0} & I_d
        \end{array} \right]
\end{align}

The matrix selects HOG features that corresponds to the first cell and the third cell. 

Using the selector $S$, we can define new vectorized HOG features $\bar{x} = S x,
\bar{\mu} = S\mu, \bar{\Sigma} = S \Sigma S^T$. After solving the resulting system (which
is now smaller than in the WHO approach) we find

\begin{equation}
    \bar{w}=\bar{\Sigma}^{-1}(\bar{x} - \bar{\mu}) \label{eq:nz-who},
\end{equation}

To speed up the convolution, we restore zero cells and reshape the vector $\bar{w}$
and compute convolution.

% Notice that $\Sigma$ is not square-rooted in Eq.~\ref{eq:nz-who}, whereas it is
% in the definition of whitening (Eq.~\ref{eq:whitening}). This is because the
% whitened template will only be used in the context of detection, which involves
% the inner product with the whitened HOG features of the query image $y$:
% \begin{equation}
% (y-\mu)^{T}\Sigma^{-\frac{1}{2}}\Sigma^{-\frac{1}{2}}(x-\mu)\\
% = (y-\mu)^{T} w.
% \end{equation}
% We can therefore capture the whitening of the query image features in $w$.

% We use the GPU to parallelize the lookup of the correct autocovariance cells during covariance matrix syntheses, which is \scream{x} times faster than using the CPU (Figure \ref{fig:covariancetime}).

%\scream{chris:We propose Non-Zero Whitened Histograms of Orientations (NZ-WHO), an improvement over WHO which is both more discriminative than WHO and can be computed many orders of magnitude faster than WHO (70ms as opposed to to several seconds for WHO). This speedup allows us to generate NZ-WHO templates on the fly, allowing us to quickly evaluate arbitrary viewpoint hypotheses.}
%\scream{add comments about the fig:whocomparison}

\begin{figure}[t]
  \begin{center}
%   \fbox{\rule{0pt}{2in} \rule{0.9\linewidth}{0pt}}
    % include whitening all centered cells
    % \includegraphics[width=0.32\linewidth]{whiten_all_crop}
    \setlength\tabcolsep{3pt}
    \begin{tabular}{ccc}
      HOG & WHO & NZ-WHO \\
%     \begin{turn}{90}$w_+$\end{turn} &
    \includegraphics[width=0.28\linewidth]{hog_crop} &
    \includegraphics[width=0.28\linewidth]{whiten_all_crop} &
    \includegraphics[width=0.28\linewidth]{whiten_non_zero_crop} \\
    % include whitening all centered cells
     % \includegraphics[width=0.282\linewidth]{whiten_all_neg_crop} 
%     \begin{turn}{90}$-w_-$\end{turn} &
     \includegraphics[width=0.28\linewidth]{hog_neg_crop} &
     \includegraphics[width=0.28\linewidth]{whiten_all_neg_crop}  &
     \includegraphics[width=0.28\linewidth]{whiten_non_zero_neg} \\
    % include whitening all centered cells
 %    \begin{turn}{90}ihog\cite{vondrick2013}\end{turn} &
    % \cite{vondrick2013}& 
     \includegraphics[width=0.28\linewidth]{ihog_hog200_crop.png} &
     \includegraphics[width=0.28\linewidth]{ihog_whiten_all200_crop.png} &
     \includegraphics[width=0.28\linewidth]{ihog_whiten_non_zero200_crop.png} \\
 \end{tabular}
  \end{center}
  \caption{Comparison of HOG, WHO and NZ-WHO. Visualization of positive weights (first row),  visualization of negative weights (second row), HOGgles \cite{vondrick2013} (third row). Note that for WHO, whitening all cell result in strong negative edges on the empty region}
  \label{fig:whocomparison}
\end{figure}


\subsection{Fast Whitening using Conjugate Gradient}
\label{sec:fastwhiten}
% To speed up LDA synthesize, we introduced Conjugate Gradient method.
% for solving the matrix inversion Eq.~\ref{eq:nz-who}.
% This process speeds up , making it applicable to an on-the-fly rendering setting.
% It originates from the insight that an iterative Conjugate Gradient method can be used to whiten a HOG template, which is much faster than the original procedure proposed in~\cite{Hariharan12} based on decomposition.

Synthesizing an LDA template Eq.~\ref{eq:nz-who} requires solving the system of
linear equations, $\bar{\Sigma} \bar{w} = (\bar{x} - \bar{\mu})$. In
\cite{Hariharan12}, the authors make use of the fact that covariance matrices
are symmetric and positive semidefinite to solve the system via the Cholesky
decomposition with Gaussian Elimination, which requires $O(n^3)$ time.

The Conjugate Gradient method is an iterative algorithm for solving symmetric
positive definite systems which runs in $O(n^2\kappa)$ time, where $\kappa$ is
the condition number of the matrix. \cite{Shewchuk94}.
%
This makes Conjugate Gradient faster than decomposition for matrices with small
condition numbers relative to their size.

The covariance matrix for HOG templates is typically
ill-conditioned\cite{Hariharan12}, but adding a small regularization constant to
the diagonal reduces its condition number.
We use a constant of $0.15$, which reduces the condition number from $10^{20}$
to $50$, much smaller than the dimension of the matrix (7000).

As a result, a GPU implementation of conjugate gradient converges in 60
ms when using 250 HOG cells on Nvidia GTX660, two orders of magnitude faster
than the using Cholesky factorization with Gaussian Elimination.

We report the real time analysis of whitening using decomposition and
conjugate gradient methods in Fig.~\ref{fig:whotime}. (a) compares the
absolute runtime of the different methods while (b) gives the obtained
speedup. We see that % We compare the speed of each of template generation methods in
% Fig.~\cite{fig:covariancetime_crop} Using naive Cholesky decomposition,
WHO template takes several seconds for realistic template sizes of
several hundred cells, while using the iterative Conjugate Gradient
method, it only takes 60ms. If we use NZ-WHO, we can gain extra speed
up since we only whitens non-zero cells.

In addition, since the iterative Conjugate Gradient method directly
tries to reduce the residual (the norm of $y-Ax$ for $Ax = y$), it is
more numerically stable than Cholesky decomposition with Gaussian
Elimination. We vary the number of cells in a template and show that
the residual of NZ-WHO is smaller than WHO Fig.~\ref{fig:whoresidual}.
% Also, the residual from the Conjugate Gradient method (the norm of $y-Ax$ for $Ax = y$) is smaller than 
% that of Cholesky decomposition. 

\begin{figure}[t]
  \begin{center}
  \begin{tabular}{cc}
     \includegraphics[width=0.5\linewidth]{whotime} & 
     \includegraphics[width=0.5\linewidth]{speedup}\\
     (a) & (b) \\
 \end{tabular}
  \end{center}
  \caption{(a) Runtime analysis of whitening. HOG means feature
    extraction time, WHO-Chol uses ~\cite{Hariharan12} and
    WHO-CG uses iterative Conjugate Gradient. NZ-WHO-CG (Ours) uses only
    non-zero cells and Conjugate Gradient. (b) final speedup
    of NZ-WHO vs. WHO.}
  \label{fig:whotime}
\end{figure}
%
\begin{figure}[t]
  \centering
  \includegraphics[width=0.5\linewidth]{residual}
  \caption{Residuals from different method}
  \label{fig:whoresidual}
\end{figure}


\subsection{High Resolution Templates and FFT based Convolution}
\label{sec:fft} 

We generate high resolution templates with more than 250 HOG cells to capture
details of an object to give accurate 2D-3D matching. These large templates
cause computational burden when computing convolution. Though good for
detecting accurate model and pose, convolution of high-resolution templates are
much slower since computation time scales linearly with the number of HOG
cells in the template. To overcome and speed up the convolution, we used FFT
based GPU convolution \cite{Podlozhnyuk} which scales. Briefly, for length $n$
signal and length $m$ filter, naive convolution takes $O(nm)$ time whereas
FFT-based convolution takes $O\left( (n + m)\log (n+m) \right)$ time. For large
$m$ (high resolution templates), we can gain computational advantage.


\section{Pose Fine-Tuning via MCMC}
\label{sec:fine}
% Estimating viewpoint continuously to fine tune a initial matching
The NZ-WHO template matching method we have presented
(Sect.~\ref{sec:nzwho}) makes template generation and evaluation
computationally inexpensive. This means that we can use a
hypothesize-and-test scheme to efficiently explore the continuous
parameter space to find the best object pose, scale, 3D CAD model type
and camera focal length.
%
In particular, we propose to implement this parameter search as a
Markov Chain Monte Carlo (MCMC) procedure based on the
Metropolis-Hastings algorithm.

\paragraph{Probabilistic formulation.}
We parameterize the continuous parameter space as  $x = [p, m, f]$,
where $p$ is the 3D rotation of the CAD model, $m$ is the discrete CAD
model index, and $f$ is the focal length.

We model the probability that the object in the test image
$\mathcal{I}$ is generated by the parameters $x$ as a distribution in
the exponential family, and let
\begin{align}
    P(x) & \sim e^{ c_4 \max_{s} w(x) \ast \mathcal{T}_s(\mathcal{I})},
\end{align}
where $\max_{s} w(x) \ast T(\mathcal{I})$ is the convolution score of
template $w(x)$ with image feautures $T(\mathcal{I})$, as defined
in Sect.~\ref{}.

\paragraph{Inference.}
We approximate the MAP solution for $x$ by drawing samples from the
distribution $P(x)$, using the Metropolis-Hastings
algorithm. Specifically, we use a variant that changes only a single
component of the parameter vector $x$ at a time, termed Single
Component Metropolis Hastings.

As proposals

For proposal distribution, we used Gaussian distribution to propose next viewpoint and for viewpoint proposals
and uniform distribution for model change proposals.  


\begin{align}
    A(x \rightarrow x') & =  min\left( 1,  \frac{P(x' g(x') \rightarrow x)}{P(x) g(x \rightarrow x')}\right) \\
    g(x \rightarrow x + p_i') & \sim \mathcal{N}(\mu = 0,\sigma = c_1) \quad i \in \{1,2,3\}\\
    g(x \rightarrow x + f') & \sim \mathcal{N}(\mu = 0,\sigma = c_2)\\
    g(x \rightarrow x + m') & \sim c_3 \delta(m) + (1-c_3) Unif(1,M)
\end{align}


Where $g$ is the proposal distribution and $P$ is the distribution of which we
defined using Gaussian distribution around previous state. $A$ is the
acceptance probability where we accept new proposal with $A(x \rightarrow x')$.


$P$ is the distribution of the pose and location of an object on image
$\mathcal{I}$.

One can also think of the image as an image cropped by detection
bounding box. We synthesize covariance matrix and create NZ-WHO
template $w(x)$ for particular viewpoint, focal length and CAD model.



In sum, for new proposal $x'$, we make a NZ-WHO template on the fly 
We found out that it requires good initialization to converge to local minima. 


% there are several advantages that was not possible such as
% \textbf{on-the-fly} template generation and making large number of
% templates
\begin{figure}[t]
\centering
    \includegraphics[width=0.7\linewidth]{tuning2} \\ [-5pt]
    \caption{Different modes of Metropolis Hasting proposals, see Section \ref{sec:fine}}
 \label{fig:tuningmode}
\end{figure}


\paragraph{Initialization.}
To fine-tune continuous pose and location of object, we require strong
initialization. The initialization is a rough bounding box with small context
and viewpoint. Since most of the standard detectors does not provide viewpoint
of an object, we first run an ensemble of NZ-WHO templates to get rough
viewpoints. Once we get rough viewpoint, we initialize the proposal $x_0$ and
start sampling proposals.

We initialize the $x$ using the result from the previous stage where
we run our ensemble or NZ-WHO templates to get rough estimate.

\section{Experiments}
\label{sec:experiments}
%
In this section, we give an experimental evaluation of our approach,
highlighting three different aspects.
%
First, we verify that our NZ-WHO method delivers performance that is
at least on par with the original WHO formulation~\cite{Hariharan12} in terms of
accuracy, while at the same time resulting in large computational
savings (Sect.~\ref{sec:variations}).
%
Second, we demonstrate that our method can be used for multi-view
object class detection in isolation. It can be applied in a sliding
window fashion and deliver 2D bounding box as well as viewpoint
information. Our method compares on par to state-of-the-art in this
case (Sect.~\ref{sec:exp_iso}).
%
And third, we show that our method can be used to complement the
detections provided by an existing object class detector, such
as DPM~\cite{Felzenszwalb10} or RCNN~\cite{Girshick14}. In this case,
we show a considerable performance improvement compared to previous
work in the task of joint object class detection and VP
estimation (Sect.~\ref{sec:exp_enrich}).

\paragraph{Setup.}
We use established benchmark datasets to validate our approach, namely
the 3D Object Classes dataset~\cite{Savarese07}, and
PASCAL3D+~\cite{Xiang14}, a recently proposed extension of Pascal
VOC'12~\cite{pascal12} that provides additional annotations in the
form of aligned 3D CAD models. In both cases, we use the test data
provided by the respective datasets, but train our models entirely
from rendered 3D CAD model images.

% In this section, we validate our proposed NZ-WHO templates in two
% different ways. First, we make a set of NZ-WHO templates for a set of
% CAD models from various viewpoints. Then we run the bank of NZ-WHO
% templates as ensembles of weak detectors and show that the rendering
% image can be We want to see how robust the NZ-WHO to real images a
%
% Second, we show that our method can complement the detection bounding
% boxes coming out of standard object detectors. This process will
% enrich object detection bounding boxes with 2D-3D matching which gives
% continuous viewpoint estimation, a CAD model (rendering, depth) and
% fine-grained category.  

\subsection{WHO Variants}
\label{sec:variations}
To validate our approach, we run an ensemble of NZ-WHO templates as a
bank of exemplar detectors and compare its performance with other WHO
variants, notable WHO~\cite{Hariharan12} and the original non-whitened
HOG~\cite{Dalal05}. In addition, we evaluate WHO-CG and WHO-CG-Z:
WHO-CG uses iterative Conjugate Gradient method to generate
WHO. WHO-CG-Z whitens the whole template, but zeros out textureless
region. NZ-WHO-CG is the NZ-WHO which whitens only non-zero cells using
iterative Conjugate Gradients  (our method).

Tab.~\ref{tab:who_initializations} gives the corresponding results for
the various methods on a subset of the 3D Object Classes car
dataset~\cite{Savarese07} (all images corresponding to one particular
car instance, seen from different viewpoints), reporting 3 quantities:
detection performance in average precision (AP), pose estimation in mean
precision in pose estimation (MPPE), and the respective runtime.
%
For all methods, the table gives the results with and without
calibration, using the calibration method of~\cite{Aubry14}. This
calibration learns affine transformation of the detection confidence.
For each method, we generated templates for 1 CAD model exemplar
rendered from 24 azimuth and 4 elevation angles.
% Since WHO variants uses HOG features, they are slower than
% HOG templates but WHO variants in general performs better than WHO by
% significant margin. However, WHO using Cholesky decomposition with
% Gaussian Elemination takes the most time yet performs worse than
% NZ-WHO which is two order of magnitude faster to generate.

\paragraph{Results.}
In Tab.~\ref{tab:who_initializations}, we observe that, on average,
calibration indeed improves performance in terms of AP, sometimes
drastically (e.g., from $54.4$ to $92.8$ for WHO-CG-Z), as observed in
prior work~\cite{Aubry14}. At the same time, calibration is
computationally expensive, resulting in generation time that are two orders of
magnitudes larger than without calibration. Strikingly, our method
NZ-WHO-CG achieves the second best AP of $90.0$ while completing in 79
ms vs. 8.7 s when using calibration.

% We empirically found out that NZ-WHO performs reasonably well without time
% consuming calibration stage compare to other variational method to generate
% templates. We used 1 CAD model with viewpoints covering 24 azimuths and 4
% elevations. Each template takes approximately 80 milliseconds to generate a
% NZ-WHO template. We also calibrated templates using the method presented in
% \cite{Aubry14}. e.% Note that in Section \ref{experiments}, we provide the same
% % experiment with 9 CAD models.


\begin{table}[!htbp]
  \footnotesize
  \setlength{\tabcolsep}{1pt}
  \centering
  \begin{tabular}{|c|c|r|c|r|}
    \hline
    Methods (AP/MPPE) & before calib.  & time & after calib.~\cite{Aubry14} & time \\
    \hline\hline
    HOG\cite{Dalal05}      & 72.3 / 65.0          &  31ms  & 60.4 / 50.2           & 8.7 sec \\ 
    WHO\cite{Hariharan12}  & 82.1 / \textbf{85.4} &  6162ms& 84.4 / 83.0           & 15.4 sec\\
    WHO-CG                 & 81.7 / 84.9          &  104ms & 83.7 / \textbf{87.3}  & 8.3 sec \\
    WHO-CG-Z               & 54.4 / 65.1          &  103ms & \textbf{92.8} / 86.7  & 8.7 sec \\
    % NZC-WHO-Z    & 89.10 /\textbf{78.64} &    & 91.15 / 74.79                  &     \\
    NZ-WHO-CG {\em (ours)} & \textbf{90.0} / 82.8 &   79ms & 90.3 / 86.8           & 8.5 sec \\
    \hline
  \end{tabular}
  \caption{Average Precision(AP), Mean Precision in Pose Estimation
    (MPPE)~\cite{Lopez-Sastre11}, and generation time for variations of WHO on
    3D Object Classes cars~\cite{Savarese07}. Please see text for
    details.}
  % WHO refers to standard WHO using the
  % method presented in \cite{Hariharan12}, WHO-CG uses iterative
  % Conjugate Gradient method to generate WHO. WHO-CG-Z uses whiten the
  % whole template and zero out textureless region. NZ-WHO-CG is the
  % NZ-WHO which whitens only non-zero cells using iterative Conjugate
  % Gradient method. The time column indicates the time to generate one
  % template. We followed calibration procedure presented in \cite{Aubry14}.}
  \label{tab:who_initializations}
\end{table}


\subsection{2D-3D Matching as an Object Detector} 
\label{sec:exp_iso}
In this section, we evaluate the performance of our method in
isolation, for the task of object class detection and viewpoint
estimation, on the 3D Object Classes dataset~\cite{Savarese07}, for the
categories {\em car} and {\em bicycle}.

To that end, we create an ensemble of NZ-WHO templates (Sect.~\ref{sec:approach})
using $9$ different CAD models and a total of 192 different
viewpoints, using 4 different elevation angles, 24 azimuth angles,
and 2 focal lengths. We run the entire ensemble exhaustively over
each test image, in a sliding window fashion.

 Performance is measured in Average Precision
(AP) for object detection and Mean Precision in Pose Estimation
(MPPE)~\cite{Lopez-Sastre11}. Pose estimation is here understood as a
discrete problem in which the predicted azimuth angle is binned into a
set of $8$ discrete viewpoint classes.
% Instance-level 2D-3D matching (registration) is a difficult problem
% since we need a CAD model that looks similar to the object in the
% image. Especially, due to extreme intraclass variability of chair
% \cite{Aubry14} used a very large collection of CAD models to cover
% variety of chairs. Similarly, \cite{Lim14}

\paragraph{Results.}
Tab.~\ref{tab:3dobject} gives the corresponding results, comparing our
method to two recent state-of-the-art baselines, the aspect layout
model (ALM~\cite{Xiang12}), and the
DPM-VOC+VP~\cite{Pepik12}. Fig.~\ref{fig:3dobject_vis} gives
qualitative results.

We observe that our model performs on par with the state-of-the-art
methods in terms of AP ($99.8$) for cars. It performs slightly worse
on bicycles than DPM-VOC+VP ($93.0$ vs. $98.8$), but on par with the
ALM ($93.0$). In viewpoint estimation, our model performs slightly
worse than both methods ($91.7$ vs. $93.4$ and $97.5$ for cars, and
$90.9$ vs. $91.4$ and $97.5$ for bicycles, respectively).

This result is encouraging, since our approach reaches a level of
performance that is on par to current state-of-the-art while at the
same time being much faster to train. It takes merely a few minutes to
train, while both ALM~\cite{Xiang12} and DPM-VOC+VP~\cite{Pepik12} are
complex models that optimize non-convex objective functions during
training, which is only made tractable by resorting to delayed
constraint generation in the form of hard negative mining, and can
easily take a day on a single machine. In addition, our approach uses
only rendered images, avoiding the need for real-world training data
with costly bounding box and viewpoint annotations.
%
\begin{table}[!htbp]
    \footnotesize
  \begin{center}
    \begin{tabular}{|c|c|c|c|}
    \hline
     AP/MPPE& Ours & ALM\cite{Xiang12} & VOC-DPM\cite{Pepik12} \\
    \hline\hline
    car     & \textbf{99.8} / 91.7 & 98.4 / 93.4 & \textbf{99.8} / \textbf{97.5} \\ 
    bicycle & 93.0 / 90.9          & 93.0 / 91.4 & \textbf{98.8} / \textbf{97.5} \\
    \hline
    \end{tabular}
  \end{center}
  \caption{Average Precision (AP) and Mean Precision in Pose
    Estimation (MPPE) on 3D Object Classes~\cite{Savarese07} cars.}% Our method produces high quality 2D-3D matching yet performs on par with state-of-the art detectors.}
  \label{tab:3dobject}
\end{table}

\begin{figure}[h]
    \begin{center}
    \setlength\tabcolsep{0pt}
\begin{tabular}{cc}
  \includegraphics[width=0.35\linewidth]{car_3dobject/7.png} &   
  \includegraphics[width=0.35\linewidth]{car_3dobject/8.png}\\ [-15pt]
%   (a) & (b) \\[0pt]
  \includegraphics[width=0.35\linewidth]{car_3dobject/11.png} &   
  \includegraphics[width=0.35\linewidth]{car_3dobject/12.png}\\ [-5pt]
%  (c) & (d) \\[0pt]
\end{tabular}
\end{center}
\caption{Detection results on 3D Object
  Classes~\cite{Savarese07}. Original image (left) and detection
  result overlaid on top (right).}% with a bounding box and corresponding confidence score (right).}
  \label{fig:3dobject_vis}
\end{figure}


% Also, we evaluated our method on PASCAL 2007 dataset.
% 
% \begin{table}[!htbp]
%   \begin{center}
%   \begin{tabular}{|c|c|c|c|c|}
%     \hline
%      AP &    Ours &    ELDA \cite{Hariharan12} & ESVM \cite{Malisiewicz11} & VOCDPM \cite{Pepik12} \\
%     \hline\hline
%     car      & 27.          &  38.4  & 40.1    &65.7  \\ 
%     bicycle  & 43.74 & 41      &  40.7   & 61.3   \\
%     \hline
%     \end{tabular}
%   \end{center}
%   \caption{Average Precision(AP) on PASCAL 2007 dataset. Though we do not need any training images and made templates in few minues, it performs on par with state of the art detector yet it gives high quality 2D-3D matching.}
%   \label{tab:pascal07}
% \end{table}

\subsection{Enriching Existing Detections}
%\subsection{Fine-Tuning and Viewpoint Estimation}
\label{sec:exp_enrich}
In this section, we use our method to enrich the detections provided
by existing, high-performance object detectors with additional output
in the form a 3D pose, focal length, and 3D CAD model exemplar shape.
% High performance object detectors usually generalizes very well. It
% can handle various intraclass variability, viewpoint change and
% occlusion. However due to this high generalization, sometimes it is
% very difficult to get accurate pose, category or 2D-3D matching. Our
% method, specialized in giving such accurate information, can be
% applied on any generic object detectors to give high quality metadata.

To show such ability, we evaluate our method on the PASCAL3D+
dataset~\cite{Xiang14}. This dataset augments PASCAL 2012 images with
high quality viewpoint annotations thus is ideal to measure pose
estimation. The dataset proposes new metric called Average Viewpoint
Precision (AVP) where it measures the area under viewpoint precision
and detection recall curve. The viewpoints is measured by azimuth
similarity. If the distance between predicted azimuth and ground truth
azimuth is below a certain threshold, the viewpoint is correct.
The baseline methods V-DPM~\cite{Xiang14} and
DPM-VOC+VP~\cite{Pepik12}) reported on the PASCAL3D+ dataset are
variants of DPM~\cite{Felzenszwalb10} where each component of DPM
account for various azimuths. Thus V-DPM and VOC-DPM provide discrete
azimuth only whereas our method provides 3D viewpoint (yaw, pitch,
roll), CAD instance (model index, rendering, depth) and focal length.
%
\begin{figure}[t]
 \begin{center}
    \setlength\tabcolsep{0pt}
    \begin{tabular}{ccc}
    % \includegraphics[width=0.9\linewidth]{tuning} 
   \includegraphics[width=0.33\linewidth]{tuning/1.png} &
   \includegraphics[width=0.33\linewidth]{tuning/2.png} &
   \includegraphics[width=0.33\linewidth]{tuning/3.png} \\[-5pt]
%   (a) & (b) & (c)\\
   \includegraphics[width=0.33\linewidth]{tuning/4.png} &
   \includegraphics[width=0.33\linewidth]{tuning/5.png} &
   \includegraphics[width=0.33\linewidth]{tuning/6.png} \\[-5pt]
%   (d) & (e) & (f)\\
   \end{tabular}
 \end{center}
 \caption{Effect of fine tuning. (left) original image, (middle) initial detection, (right) continuous fine tuning using Single-Component Metropolis Hastings}
 \label{fig:tuning}
\end{figure}

\paragraph{Setup.}
We use detection bounding boxes provided by both object detectors and
augment it using our method to give viewpoint estimation. We use
detection bounding boxes from two different methods:
DPM-VOC+VP\cite{Pepik12} and R-CNN~\cite{Girshick14}, both in their
variants trained on 8 viewpoint categories, since these perform best
in terms of AP. For both cases,
we use the original score for plotting precision-recall curves
(meaning that we can not improve over their AP), but
enrich the detections with additional viewpoint information provided
by our method. We fall back to a default
viewpoint prediction ($0$ angles) in case the confidence of our
method falls below a threshold.

For both cases, we compare the performance of a discrete incarnation
of our method (Sect.~\ref{sec:approach}) and our full model, including the
fine-tuning based on MCMC (Sect.~\ref{sec:fine}).

Quantitative results are given in Tab.~\ref{tab:pascal12} in terms of
AP and AVP and in Fig.~\ref{fig:car_cnn_ap} in terms of
precision-recall plots and viewpoint confusion matrices. Example
outputs of our method applied to candidate object detections are given
in Fig.~\ref{fig:pascal12cnn}, and the effect of fine-tuning is
visualized in Fig.~\ref{fig:tuning}. More qualitative results are part of
supplemental material
\paragraph{Results.}
%
In Tab.~\ref{tab:pascal12}, we make mainly two observation. First, we
see that adding our method to DPM-VOC+VP consistently improves
performance in terms of AVP, for both car and bicycle, and across all
viewpoint bins. This is already the case for our discrete method: the
improvement ranges from $0.3$ for bicycle-8v to $5.8$ for car-4v. 
%
Using the R-CNN as the base detector increases AVP even more, in
particular for the bicycle class: for bicycle-4v, our discrete method
improves over the corresponding DPM-VOC+VP result by $12.6$.

The second observation is that the fine-tuning based on MCMC can
indeed improve pose estimation performance slightly, e.g., by $1.1$
for bicycle-16v and using the DPM-VOC+VP as the base detector, or by
$1.1$ for bicycle-16v when using the R-CNN as the base detector.
%
In Fig.~\ref{fig:tuning}, we visualize the effect of fine-tuning
qualitatively for two different test images. In both cases, the
fine-tuned visually comes closer to the true 3D pose.
% We got significant improvement on viewpoint estimation by using our
% method on top of general purpose detectors. Note that our method
% estimates azimuth, elevation, in-plane rotation, focal length
% estimation, fine-grained categorization whereas V-DPM, VOC-DPM only
% estimate discrete viewpoints.

\begin{figure*}[t]
\setlength\tabcolsep{1pt}
\centering
\begin{tabular}{|ccccc|}
   \hline
  \includegraphics[width=0.24\textwidth]{car_cnn/1a.png} &   
  \includegraphics[width=0.24\textwidth]{car_cnn/1b.png} &   
  \includegraphics[width=0.24\textwidth]{car_cnn/1c.png} &   
  \includegraphics[width=0.24\textwidth]{car_cnn/1d.png}  \\  
%   \hline
%  \includegraphics[width=0.24\textwidth]{car_cnn/2a.png} &   
%  \includegraphics[width=0.24\textwidth]{car_cnn/2b.png} &   
%  \includegraphics[width=0.24\textwidth]{car_cnn/2c.png} &   
%  \includegraphics[width=0.24\textwidth]{car_cnn/2e.png}  \\  
%   \hline
  \includegraphics[width=0.24\textwidth]{car_cnn/4a.png} &   
  \includegraphics[width=0.24\textwidth]{car_cnn/4b.png} &   
  \includegraphics[width=0.24\textwidth]{car_cnn/4c.png} &   
  \includegraphics[width=0.24\textwidth]{car_cnn/4d.png}  \\  
%    \hline
%   \includegraphics[width=0.24\textwidth]{car_cnn/10a.png} &   
%   \includegraphics[width=0.24\textwidth]{car_cnn/10b.png} &   
%   \includegraphics[width=0.24\textwidth]{car_cnn/10c.png} &   
%   \includegraphics[width=0.24\textwidth]{car_cnn/10d.png}  \\  
%   \hline 
%   \includegraphics[width=0.24\textwidth]{bicycle_cnn/3a.png} &   
%   \includegraphics[width=0.24\textwidth]{bicycle_cnn/3b.png} &   
%   \includegraphics[width=0.24\textwidth]{bicycle_cnn/3c.png} &   
%   \includegraphics[width=0.24\textwidth]{bicycle_cnn/3d.png}  \\
%
% sorry no space
%  \hline
%  \includegraphics[width=0.24\textwidth]{bicycle_cnn/7a.png} &   
%  \includegraphics[width=0.24\textwidth]{bicycle_cnn/7b.png} &   
%  &\\
%  \hline
  \includegraphics[width=0.24\textwidth]{bicycle_cnn/4a.png} &   
  \includegraphics[width=0.24\textwidth]{bicycle_cnn/4b.png} &   
  \includegraphics[width=0.24\textwidth]{bicycle_cnn/4c.png} &   
  \includegraphics[width=0.24\textwidth]{bicycle_cnn/4d.png}  \\
  \hline
  CNN Proposals & \multicolumn{3}{|c|}{Our matching results on proposal bounding boxes} \\
  \hline
\end{tabular}
\caption{Example enriched bounding boxes. Given R-CNN\cite{Girshick14}
  detection bounding boxes, our method predicted 2D-3D matching
  reasonably. On the first column, R-CNN detection bounding boxes
  overlaid. From the second to the last columns, output of our method
  given a R-CNN box. Blue boxes are R-CNN output and purple boxes are
  the tightest bounding box enclosing predicted CAD model.}
  \label{fig:pascal12cnn}
\end{figure*}
%
\begin{table*}[t]
	\scriptsize
  \begin{center}
    \begin{tabular}{|c||c||c||c|c||c|c|}
    \hline
    AP/AVP              & V-DPM \cite{Xiang14} & DPM-VOC+VP \cite{Pepik12}  & \cite{Pepik12} + Ours (discrete) & \cite{Pepik12} + Ours (full) & R-CNN + Ours (discrete) & R-CNN + Ours (full)\\
    \hline\hline
    aeroplane-4v        & 40.0 / 34.6         & 41.5 / 37.4             & 41.1 / 32.7       & 41.1 / 30.5        & 67.8 / \textbf{41.3}         & 67.8 / 40.8 \\ \hline
    aeroplane-8v        & 39.8 / 23.4         & 40.5 / 28.6             & 41.1 / 26.8       & 41.1 / 24.2        & 67.8 / 26.8         & 67.8 / \textbf{27.4} \\ \hline
    aeroplane-16v       & 43.6 / 15.4         & 38.0 / 15.9             & 41.1 / 17.7       & 41.1 / 16.3        & 67.8 / 21.2         & 67.8 / \textbf{21.5} \\ \hline
    aeroplane-24v       & 42.2 / 8.0          & 36.0 /  9.7             & 41.1 / 10.9       & 41.1 / 10.2        & 67.8 / 15.6         & 67.8 / \textbf{15.7} \\ \hline
    \hline
    car-4v              & 37.2 / 20.2         & 45.6 / 36.9             & 47.6 / 42.7       & 47.6 / \textbf{42.7}        & 49.6 / 41.5         & 49.6 / 41.5\\ \hline
    car-8v              & 37.3 / 23.5         & 47.6 / 36.6             & 47.6 / \textbf{39.8}       & 47.6 / 39.5        & 49.6 / 38.0         & 49.6 / 39.0\\ \hline
    car-16v             & 36.6 / 18.1         & 46.0 / 29.6             & 47.6 / 32.7       & 47.6 / 33.0        & 49.6 / 34.0         & 49.6 / \textbf{34.3}\\ \hline
    car-24v             & 36.3 / 13.7         & 42.1 / 24.6             & 47.6 / 27.4       & 47.6 / 27.4        & 49.6 / 27.0         & 49.6 / \textbf{27.6}\\ \hline
    \hline
    chair-4v            & 11.1 / 6.8          & 8.7  / 6.1              &  11.4 / 7.1       & 11.4 / 6.7         & 25.2 / 10.6         & 25.2 / \textbf{10.7} \\ \hline
    chair-8v            & 11.4 / 5.9          & 11.3 / 9.4              &  11.4 / 6.6       & 11.4 / 6.6         & 25.2 / \textbf{9.4}          & 25.2 / 9.3 \\ \hline
    chair-16v           & 12.8 / 6.0          & 10.2 / 6.1              &  11.4 / 4.7       & 11.4 / 4.7         & 25.2 / \textbf{6.7}          & 25.2 / 6.7 \\ \hline
    chair-24v           & 12.6 / 4.4          & 8.0  / 4.2              &  11.4 / 3.1       & 11.4 / 3.4         & 25.2 / 4.6          & 25.2 / \textbf{5.4} \\ \hline
    \hline
    bicycle-4v          & 45.2 / 41.7         & 46.9 / 43.9             & 48.1 / 47.6       & 48.1 / 46.6        & 61.7 / 56.5         & 61.7 / \textbf{56.7}\\ \hline
    bicycle-8v          & 47.3 / 36.7         & 48.1 / 40.3             & 48.1 / 40.6       & 48.1 / 40.6        & 61.7 / 48.9         & 61.7 / \textbf{49.2}\\ \hline
    bicycle-16v         & 46.5 / 18.4         & 45.6 / 22.9             & 48.1 / 26.2       & 48.1 / 27.3        & 61.7 / 34.7         & 61.7 / \textbf{35.8}\\ \hline
    bicycle-24v         & 44.4 / 14.3         & 45.9 / 16.7             & 48.1 / 21.5       & 48.1 / 20.9        & 61.7 / \textbf{27.0}         & 61.7 / 23.9\\ \hline
    \end{tabular}
  \end{center}
\caption{Average Precision(AP) and Average Viewpoint Precision (AVP) on
  PASCAL3D+~\cite{Xiang14}. For combined methods (* + Ours), we use
  bounding boxes from * and augment viewpoint using our method.}
  % R-CNN
  % and run our method to produce 2D-3D matching. If our method fails to
  % give viewpoint, it predicts the viewpoint to be 0. Note that our
  % method gives high quality metadata such as continuous 3D viewpoint,
  % CAD model (rendering depth) and fine-grained category, whereas
  % base-line methods gives 1D discrete azimuths. Our R-CNN was not an
  % optimized version so the detection AP is lower than the
  % state-of-the-art R-CNN performance}
\label{tab:pascal12}
\end{table*}

\begin{comment}
\hline
boat-4v             & 3.0 / 1.5           & 0.5 / 0.3               &  /        &  /         & 27.9 / 11.3         & 27.9 / 11.5 \\ \hline
boat-8v             & 5.8 / 1.0           & 0.5 / 0.2               &  /        &  /         & 27.9 / 5.4          & 27.9 / 5.4 \\ \hline
boat-16v            & 6.2 / 0.5           & 0.7 / 0.3               &  /        &  /         & 27.9 / 2.0          & 27.9 / 2.0 \\ \hline
boat-24v            & 6.0 / 0.3           & 5.3 / 2.2               &  /        &  /         & 27.9 / 1.8          & 27.9 / 2.5 \\ \hline
\end{comment}

\begin{figure}[h]
\setlength\tabcolsep{1pt}
\centering
\begin{tabular}{cc}
%  \hline
  \includegraphics[width=0.49\linewidth]{car_cnn4_crop.png} &   
  \includegraphics[width=0.49\linewidth]{car_cnn8_crop.png} \\   
  \includegraphics[width=0.49\linewidth]{car_cnn16_crop.png} &   
  \includegraphics[width=0.49\linewidth]{car_cnn24_crop.png} \\   
%  \hline
\end{tabular}
\caption{Average Precision (AP)(red) and Average Viewpoint Precision (AVP)(green)
  and viewpoint confusion table on PASCAL 12 car validation set using
  R-CNN + Ours (full). From the top left, 4 views, 8 views, 16 views
  and 24 views.}
 % Since our method only augment the object detection,
 %  the AP remains the same for all cases. As we increase the viewpoint
 %  estimation, the AVP decreases. Note that since we estimate null
 %  viewpoint if our method fails to find match, there are more 0
 %  viewpoint predictions (top row of confusion matrices) }
  \label{fig:car_cnn_ap}
\end{figure}

\paragraph{Robustness.}
% Generic object detectors give overlapping detection bounding boxes of the same object which can be handled by non maximal suppression. For these overlapping detections, 
While the R-CNN detector~\cite{Girshick14} is highly robust to
variations in object appearance and even occlusion and truncation,
the resulting bounding box detections vary largely in the object
portions that they contain, which provides a major challenge to any
method that uses these detections as an input for further processing,
such as ours.
%
In order to accommodate this variability, we add a considerable context
region around the proposed bounding boxes before running our method.
We assume that the object can be arbitrarily truncated by the bounding
box and search all plausible scales and translations. This can be
efficiently computed using FFT-based convolution
(Section. \ref{sec:fft}).

Fig.~\ref{fig:stability} visualizes example outputs of our method when
starting from different proposed R-CNN bounding boxes. As can be seen
from the figure, although the input bounding boxes (cyan) are often
irregular and contain truncated objects, our method reliably
generates a reasonable prediction of pose, translation, scale and CAD
model (magenta bounding boxes enclose the output of our system).

\begin{figure}[h]
\setlength\tabcolsep{5pt}
\begin{tabular}{cc}
  \includegraphics[width=0.45\linewidth]{car_pascal/1.png} &   
  \includegraphics[width=0.45\linewidth]{car_pascal/2.png} \\   
  \includegraphics[width=0.45\linewidth]{car_pascal/3.png} &   
  \includegraphics[width=0.45\linewidth]{car_pascal/4.png} \\   
\end{tabular}
\caption{Robustness of our method against irregular and truncated
  R-CNN detection proposals (cyan).}
  % \cite{Girshick14}. Blue box is the R-CNN detection that is feed into
  % the system as initial detection. Purple box and overlaid rendering
  % is the final output of the system. Note that for various overlapping
  % bounding boxes, our system accurately estimated the reasonable
  % result}
  \label{fig:stability}
\end{figure}


% For qualitative examples, we put hand-picked results from our method Fig. \ref{fig:car} and Fig. \ref{fig:bicycle}.

% \begin{figure*}[h]
% \setlength\tabcolsep{5pt}
% \begin{tabular}{|c|c|c|}
%     \hline
%   \includegraphics[width=0.325\textwidth]{car/PASCAL_car_val_init_0_Car_each_27_lim_250_lam_0_150_a_24_e_3_y_1_f_1_scale_2_00_sbin_6_level_15_skp_n_server_101_img_39.jpg} &   
%   \includegraphics[width=0.325\textwidth]{car/PASCAL_car_val_init_0_Car_each_27_lim_250_lam_0_150_a_24_e_3_y_1_f_1_scale_2_00_sbin_6_level_15_skp_n_server_101_img_78.jpg} & 
%   \includegraphics[width=0.325\textwidth]{car/PASCAL_car_val_init_0_Car_each_27_lim_250_lam_0_150_a_24_e_3_y_1_f_1_scale_2_00_sbin_6_level_15_skp_n_server_101_img_85.jpg} \\[-15pt]
% (a) & (b) & (c) \\[0pt]
% \hline
%   \includegraphics[width=0.325\textwidth]{car/PASCAL_car_val_init_0_Car_each_27_lim_250_lam_0_150_a_24_e_3_y_1_f_1_scale_2_00_sbin_6_level_15_skp_n_server_101_img_338.jpg} &   
%   \includegraphics[width=0.325\textwidth]{car/PASCAL_car_val_init_0_Car_each_27_lim_250_lam_0_150_a_24_e_3_y_1_f_1_scale_2_00_sbin_6_level_15_skp_n_server_101_img_360.jpg} & 
%   \includegraphics[width=0.325\textwidth]{car/PASCAL_car_val_init_0_Car_each_27_lim_250_lam_0_150_a_24_e_3_y_1_f_1_scale_2_00_sbin_6_level_15_skp_n_server_101_img_414.jpg} \\[-15pt]
% (d) & (e) & (f) \\[0pt]
% \hline
%   \includegraphics[width=0.325\textwidth]{car/PASCAL_car_val_init_0_Car_each_27_lim_250_lam_0_150_a_24_e_3_y_1_f_1_scale_2_00_sbin_6_level_15_skp_n_server_101_img_418.jpg} &   
%   \includegraphics[width=0.325\textwidth]{car/PASCAL_car_val_init_0_Car_each_27_lim_250_lam_0_150_a_24_e_3_y_1_f_1_scale_2_00_sbin_6_level_15_skp_n_server_101_img_1049.jpg} & 
%   \includegraphics[width=0.325\textwidth]{car/PASCAL_car_val_init_0_Car_each_27_lim_250_lam_0_150_a_24_e_3_y_1_f_1_scale_2_00_sbin_6_level_15_skp_n_server_101_img_1242.jpg} \\[-15pt]
% (g) & (h) & (i) \\[0pt]
% \hline
%   \includegraphics[width=0.325\textwidth]{car/PASCAL_car_val_init_0_Car_each_27_lim_250_lam_0_150_a_24_e_3_y_1_f_1_scale_2_00_sbin_6_level_15_skp_n_server_101_img_1244.jpg} &   
%   \includegraphics[width=0.325\textwidth]{car/PASCAL_car_val_init_0_Car_each_27_lim_250_lam_0_150_a_24_e_3_y_1_f_1_scale_2_00_sbin_6_level_15_skp_n_server_101_img_1261.jpg} & 
%   \includegraphics[width=0.325\textwidth]{car/PASCAL_car_val_init_0_Car_each_27_lim_250_lam_0_150_a_24_e_3_y_1_f_1_scale_2_00_sbin_6_level_15_skp_n_server_101_img_1452.jpg} \\[-15pt]
% (j) & (k) & (l) \\[0pt]
% \hline
%   \includegraphics[width=0.325\textwidth]{car/PASCAL_car_val_init_0_Car_each_27_lim_250_lam_0_150_a_24_e_3_y_1_f_1_scale_2_00_sbin_6_level_15_skp_n_server_101_img_1467.jpg} &   
%   \includegraphics[width=0.325\textwidth]{car/PASCAL_car_val_init_0_Car_each_27_lim_250_lam_0_150_a_24_e_3_y_1_f_1_scale_2_00_sbin_6_level_15_skp_n_server_101_img_1553.jpg} & 
%   \includegraphics[width=0.325\textwidth]{car/PASCAL_car_val_init_0_Car_each_27_lim_250_lam_0_150_a_24_e_3_y_1_f_1_scale_2_00_sbin_6_level_15_skp_n_server_101_img_1597.jpg} \\[-15pt]
% (m) & (n) & (o) \\[3pt]
% \hline
% \end{tabular}
%   \caption{Selected detections from PASCAL 07 Car categories. From the left, (1) original image, (2) overlaid true positive CAD models, (3) overlaid true positive CAD models with bounding boxes, (4) overlaid false positive CAD models with bounding boxes. Bounding box is colored from red (high confidence) to blue (low confidence) }
%   \label{fig:car}
% \end{figure*}
% \begin{figure*}[h]
% \setlength\tabcolsep{5pt}
% \begin{tabular}{|c|c|c|}
%     \hline
%   \includegraphics[width=0.325\textwidth]{bicycle/PASCAL_bicycle_val_init_0_Bicycle_each_4_lim_250_lam_0_150_a_48_e_4_y_5_f_1_scale_2_00_sbin_6_level_15_skp_n_server_102_img_459.jpg} &   
%   \includegraphics[width=0.325\textwidth]{bicycle/PASCAL_bicycle_val_init_0_Bicycle_each_4_lim_250_lam_0_150_a_48_e_4_y_5_f_1_scale_2_00_sbin_6_level_15_skp_n_server_102_img_1300.jpg} &   
%   \includegraphics[width=0.325\textwidth]{bicycle/PASCAL_bicycle_val_init_0_Bicycle_each_4_lim_250_lam_0_150_a_48_e_4_y_5_f_1_scale_2_00_sbin_6_level_15_skp_n_server_102_img_539.jpg} \\[-15pt]
% (a) & (b) & (c) \\[0pt]
% \hline
%   \includegraphics[width=0.325\textwidth]{bicycle/PASCAL_bicycle_val_init_0_Bicycle_each_4_lim_250_lam_0_150_a_48_e_4_y_5_f_1_scale_2_00_sbin_6_level_15_skp_n_server_102_img_35.jpg} &   
%   \includegraphics[width=0.325\textwidth]{bicycle/PASCAL_bicycle_val_init_0_Bicycle_each_4_lim_250_lam_0_150_a_48_e_4_y_5_f_1_scale_2_00_sbin_6_level_15_skp_n_server_102_img_83.jpg} &   
%   \includegraphics[width=0.325\textwidth]{bicycle/PASCAL_bicycle_val_init_0_Bicycle_each_4_lim_250_lam_0_150_a_48_e_4_y_5_f_1_scale_2_00_sbin_6_level_15_skp_n_server_102_img_131.jpg} \\[-15pt]
% (d) & (e) & (f) \\[0pt]
% \hline
%   \includegraphics[width=0.325\textwidth]{bicycle/PASCAL_bicycle_val_init_0_Bicycle_each_4_lim_250_lam_0_150_a_48_e_4_y_5_f_1_scale_2_00_sbin_6_level_15_skp_n_server_102_img_175.jpg} &   
%   \includegraphics[width=0.325\textwidth]{bicycle/PASCAL_bicycle_val_init_0_Bicycle_each_4_lim_250_lam_0_150_a_48_e_4_y_5_f_1_scale_2_00_sbin_6_level_15_skp_n_server_102_img_210.jpg} &   
%   \includegraphics[width=0.325\textwidth]{bicycle/PASCAL_bicycle_val_init_0_Bicycle_each_4_lim_250_lam_0_150_a_48_e_4_y_5_f_1_scale_2_00_sbin_6_level_15_skp_n_server_102_img_233.jpg} \\[-15pt]
% (g) & (h) & (i) \\[0pt]
% \hline
%   \includegraphics[width=0.325\textwidth]{bicycle/PASCAL_bicycle_val_init_0_Bicycle_each_4_lim_250_lam_0_150_a_48_e_4_y_5_f_1_scale_2_00_sbin_6_level_15_skp_n_server_102_img_322.jpg} &   
%   \includegraphics[width=0.325\textwidth]{bicycle/PASCAL_bicycle_val_init_0_Bicycle_each_4_lim_250_lam_0_150_a_48_e_4_y_5_f_1_scale_2_00_sbin_6_level_15_skp_n_server_102_img_348.jpg} &   
%   \includegraphics[width=0.325\textwidth]{bicycle/PASCAL_bicycle_val_init_0_Bicycle_each_4_lim_250_lam_0_150_a_48_e_4_y_5_f_1_scale_2_00_sbin_6_level_15_skp_n_server_102_img_375.jpg} \\[-15pt]
% (j) & (k) & (l) \\[0pt]
% \hline
%   \includegraphics[width=0.325\textwidth]{bicycle/PASCAL_bicycle_val_init_0_Bicycle_each_4_lim_250_lam_0_150_a_48_e_4_y_5_f_1_scale_2_00_sbin_6_level_15_skp_n_server_102_img_400.jpg} &   
%   \includegraphics[width=0.325\textwidth]{bicycle/PASCAL_bicycle_val_init_0_Bicycle_each_4_lim_250_lam_0_150_a_48_e_4_y_5_f_1_scale_2_00_sbin_6_level_15_skp_n_server_102_img_434.jpg} &   
%   \includegraphics[width=0.325\textwidth]{bicycle/PASCAL_bicycle_val_init_0_Bicycle_each_4_lim_250_lam_0_150_a_48_e_4_y_5_f_1_scale_2_00_sbin_6_level_15_skp_n_server_102_img_459.jpg} \\[-15pt]
% (m) & (n) & (o) \\[3pt]
% \hline
% \end{tabular}
%   \caption{Selected detections from PASCAL 12 Bike categories. From the left, (1) original image, (2) overlaid true positive CAD models, (3) overlaid true positive CAD models with bounding boxes, (4) overlaid false positive CAD models with bounding boxes. Bounding box is colored from red (high confidence) to blue (low confidence) }
%   \label{fig:bicycle}
% \end{figure*}


% \section{Object Detection with various Training Data}
% \section{Image Query}


\section{Conclusion}
\label{conclusion}
\vspace{-.2cm}
We have proposed a method for generating 3D CAD model
exemplar templates on-the-fly, based on a novel variant of
WHO features (NZ-WHO). It circumvents the need for
calibration, is computationally efficient, and allows it to be run in
an on-the-fly setting. As a result, we can use our method to enrich
existing object detections with additional information such as
precise 3D pose and 3D CAD model exemplar. Combined with an R-CNN
detector, we achieve state-of-the-art
results in joint detection and VP estimation.



\paragraph{Acknowledgement}
This work has been supported by the Max Planck Center for Visual Computing \& Communication.

\clearpage
\clearpage
{\small
\bibliographystyle{ieee}
\bibliography{egbib}
}

\end{document}
