%\subsection{Object Detection and Viewpoint Estimation}
%
Modern object detectors generalize very well, handling intraclass variability,
occlusion, truncation and viewpoint changes \cite{Felzenszwalb10, Girshick14}.
However, this generalization comes at the cost of fine-grained information,
including accurate 3D pose and object sub-category recognition. Such methods
typically produce bounding box detection hypotheses, with little further
information.

Many methods have attempted to move object detection towards richer outputs,
especially by jointly performing detection and pose estimation \cite{Pepik12,
Xiang12, Fidler12, Xiang14, Hejrati14, Aubry14, Lim14}. To achieve this,
\cite{Xiang12, Hejrati14, Fidler12} use 3D representations that deform as
viewpoint changes and \cite{Pepik12} uses geometric constraints to regularize 2D
appearance models.

The methods above perform discrete pose estimation, quantizing the viewing
sphere into a number of poses and selecting the best one during inference.
Fine-grained pose estimators, in contrast, can infer continuous (or arbitrarily
fine-grained) poses. One such method from \cite{Zia13} aligns a 3D deformable
part-based wireframe model with input images to accurately predict object poses.

More recently, \cite{Aubry14, Lim14} made progress in joint instance-level
object detection and pose estimation. To estimate pose they use synthetic
renderings of CAD models to learn discriminative mid-level patches.
\cite{Aubry14} calibrates these patches on a small set of real images, while
\cite{Lim14} present a method for learning the relative discriminativeness of
the patches.

% \textcolor{red}{Again, should we highlight the differences between these methods and ours?}

% specialized in giving such accurate information, can be applied on any generic
% object detectors to give high quality metadata.  Many Since the WHO does not
% require extensive training such as \cite{Felzenszwalb10, Malisiewicz11,
% Girshick14}, WHO is gaining momentum in 2D-3D matching. \cite{Aubry13,
% Aubry14, Lim14} combined WHO with synthetic rendering to jointly detect and
% estimate pose by making mid-level patches using WHO. Using this WSuch approach
% has an advantage over generic detections. First, it is easy to trace the
% detection to specific template. The template also has The rendering has
% corresponding CAD model and viewpoint thus enabling joint detection and
% viewpoint estimation.  To deal with textureless background \cite{Aubry13}
% whitens HOG feature and then zero out the WHO feature where there is no
% texture. But centering a HOG feature where there is no edge creates strong
% negative edges. We found out that centering only non-zero cells help detection
% but whitening all cell makes features to leak out to texture-less regions.
% With increasing popularity of rendering image, whitening, traditional signal
% processing technique, as a preprocessing stage has been widely used also in
% computer vision community.  Many object detectors \cite{Felzenszwalb10,
% Malisiewicz11} \subsection{Mid-level Patches for 2D-3D matching} FPM
% introduction : per instance matching  Seeing 3D chair : mid-level patches
  
%\begin{comment}
%  \subsection{Contributions}
 % \begin{enumerate} \itemsep1pt \parskip0pt \parsep0pt
  %  \item Introduces Non-Zero Whitened Histograms of Orientations (NZ-WHO) for real-time template generation
   % \item Continuous viewpoint and focal length tuning using on-the-fly template generation
%    \item Jointly optimizes translation, 3D viewpoint, CAD models and focal length
%     \begin{comment}
%       \item HOG Whitening procedure that do not require calibration stage
%     \end{comment}
%  \end{enumerate}
% \end{comment}
% All the code and CAD models is publicly available online \cite{Choy14} (currently not disclosed for anonymous review, visualizations are available in the supplementary material).

